\documentclass{article}\usepackage[]{graphicx}\usepackage[]{color}
%% maxwidth is the original width if it is less than linewidth
%% otherwise use linewidth (to make sure the graphics do not exceed the margin)
\makeatletter
\def\maxwidth{ %
  \ifdim\Gin@nat@width>\linewidth
    \linewidth
  \else
    \Gin@nat@width
  \fi
}
\makeatother

\definecolor{fgcolor}{rgb}{0.345, 0.345, 0.345}
\newcommand{\hlnum}[1]{\textcolor[rgb]{0.686,0.059,0.569}{#1}}%
\newcommand{\hlstr}[1]{\textcolor[rgb]{0.192,0.494,0.8}{#1}}%
\newcommand{\hlcom}[1]{\textcolor[rgb]{0.678,0.584,0.686}{\textit{#1}}}%
\newcommand{\hlopt}[1]{\textcolor[rgb]{0,0,0}{#1}}%
\newcommand{\hlstd}[1]{\textcolor[rgb]{0.345,0.345,0.345}{#1}}%
\newcommand{\hlkwa}[1]{\textcolor[rgb]{0.161,0.373,0.58}{\textbf{#1}}}%
\newcommand{\hlkwb}[1]{\textcolor[rgb]{0.69,0.353,0.396}{#1}}%
\newcommand{\hlkwc}[1]{\textcolor[rgb]{0.333,0.667,0.333}{#1}}%
\newcommand{\hlkwd}[1]{\textcolor[rgb]{0.737,0.353,0.396}{\textbf{#1}}}%
\let\hlipl\hlkwb

\usepackage{framed}
\makeatletter
\newenvironment{kframe}{%
 \def\at@end@of@kframe{}%
 \ifinner\ifhmode%
  \def\at@end@of@kframe{\end{minipage}}%
  \begin{minipage}{\columnwidth}%
 \fi\fi%
 \def\FrameCommand##1{\hskip\@totalleftmargin \hskip-\fboxsep
 \colorbox{shadecolor}{##1}\hskip-\fboxsep
     % There is no \\@totalrightmargin, so:
     \hskip-\linewidth \hskip-\@totalleftmargin \hskip\columnwidth}%
 \MakeFramed {\advance\hsize-\width
   \@totalleftmargin\z@ \linewidth\hsize
   \@setminipage}}%
 {\par\unskip\endMakeFramed%
 \at@end@of@kframe}
\makeatother

\definecolor{shadecolor}{rgb}{.97, .97, .97}
\definecolor{messagecolor}{rgb}{0, 0, 0}
\definecolor{warningcolor}{rgb}{1, 0, 1}
\definecolor{errorcolor}{rgb}{1, 0, 0}
\newenvironment{knitrout}{}{} % an empty environment to be redefined in TeX

\usepackage{alltt}
\usepackage[sc]{mathpazo}
\renewcommand{\sfdefault}{lmss}
\renewcommand{\ttdefault}{lmtt}
\usepackage[T1]{fontenc}
\usepackage{geometry}
\geometry{verbose,tmargin=2.5cm,bmargin=2.5cm,lmargin=2.5cm,rmargin=2.5cm}
\setcounter{secnumdepth}{2}
\setcounter{tocdepth}{2}
\usepackage[unicode=true,pdfusetitle,
 bookmarks=true,bookmarksnumbered=true,bookmarksopen=true,bookmarksopenlevel=2,
 breaklinks=false,pdfborder={0 0 1},backref=false,colorlinks=false]
 {hyperref}
\hypersetup{
 pdfstartview={XYZ null null 1}}

\makeatletter
%%%%%%%%%%%%%%%%%%%%%%%%%%%%%% User specified LaTeX commands.
\renewcommand{\textfraction}{0.05}
\renewcommand{\topfraction}{0.8}
\renewcommand{\bottomfraction}{0.8}
\renewcommand{\floatpagefraction}{0.75}

\makeatother
\IfFileExists{upquote.sty}{\usepackage{upquote}}{}
\begin{document}








The results below are generated from an R script.

\begin{knitrout}
\definecolor{shadecolor}{rgb}{0.969, 0.969, 0.969}\color{fgcolor}\begin{kframe}
\begin{alltt}
\hlcom{#################################################}
\hlcom{# R-for-Epi}
\hlcom{# Epidemiologic Methods II (PHW250F, PHW250G)}
\hlcom{# created by Jade Benjamin-Chung}

\hlcom{# Solutions: Homework 2, Prevalence}
\hlcom{#################################################}
\hlcom{# Load okR autograder}
\hlkwd{source}\hlstd{(}\hlstr{'setup/autograder-setup/hw2_prev/hw2_prev.ok.R'}\hlstd{)}
\end{alltt}


{\ttfamily\noindent\itshape\color{messagecolor}{\#\# here() starts at /Users/Nolan/Desktop/grading-temp/hw-prev-trial copy}}

{\ttfamily\noindent\color{warningcolor}{\#\# Warning: package 'checkr' was built under R version 3.5.2}}

{\ttfamily\noindent\color{warningcolor}{\#\# Warning: package 'assertthat' was built under R version 3.5.2}}

{\ttfamily\noindent\color{warningcolor}{\#\# Warning: package 'dplyr' was built under R version 3.5.2}}

{\ttfamily\noindent\itshape\color{messagecolor}{\#\# \\\#\# Attaching package: 'dplyr'}}

{\ttfamily\noindent\itshape\color{messagecolor}{\#\# The following objects are masked from 'package:stats':\\\#\# \\\#\#\ \ \ \  filter, lag}}

{\ttfamily\noindent\itshape\color{messagecolor}{\#\# The following objects are masked from 'package:base':\\\#\# \\\#\#\ \ \ \  intersect, setdiff, setequal, union}}\begin{alltt}
\hlkwd{AutograderInit}\hlstd{()}

\hlcom{#################################################}
\hlcom{# Read in the data and view the data}
\hlcom{#################################################}
\hlcom{# Load the dplyr package}
\hlkwd{library}\hlstd{(dplyr)}

\hlcom{# This assignment uses data from the WASH Benefits}
\hlcom{# Bangladesh trial. The trial assessed whether }
\hlcom{# water, sanitation, handwashing, and nutrition }
\hlcom{# interventions delivered separately or together}
\hlcom{# could reduce child diarrhea and/or improve child}
\hlcom{# growth. The trials used a cluster- and block-randomized }
\hlcom{# design. Within each geographic block, 8 village clusters}
\hlcom{# were randomized to a treatment or conrtol arm. }
\hlcom{# See Luby et al. 2018 for full details }
\hlcom{# (doi: http://dx.doi.org/10.1016/)}

\hlcom{# In this problem set we will calculate the prevalence}
\hlcom{# of diarrhea in different treatment arms after the }
\hlcom{# interventions were delivered. }

\hlcom{# The data and codebooks are publicly available here: }
\hlcom{# https://osf.io/pqzj5/}

\hlcom{# Load the diarrhea dataset:}

\hlstd{d} \hlkwb{=} \hlkwd{read.csv}\hlstd{(}\hlkwd{paste0}\hlstd{(here}\hlopt{::}\hlkwd{here}\hlstd{(),}\hlstr{"/data/washb-data/washb-bangladesh-diar-public.csv"}\hlstd{))}

\hlcom{# Load the dataset with treatment variables: }
\hlstd{tr} \hlkwb{=} \hlkwd{read.csv}\hlstd{(}\hlkwd{paste0}\hlstd{(here}\hlopt{::}\hlkwd{here}\hlstd{(),}\hlstr{"/data/washb-data/washb-bangladesh-tr-public.csv"}\hlstd{))}

\hlcom{# Next let's merge the two datasets together.}
\hlcom{# This will allow us to calculate the prevalence}
\hlcom{# of diarrhea in different treatment arms }
\hlcom{# (e.g., water, sanitation, handwashing, etc.)}

\hlstd{d_tr} \hlkwb{=} \hlkwd{left_join}\hlstd{(d, tr,} \hlkwc{by}\hlstd{=}\hlkwd{c}\hlstd{(}\hlstr{"block"}\hlstd{,}\hlstr{"clusterid"}\hlstd{))}

\hlcom{# Now let's filter to only keep the rows for diarrhea}
\hlcom{# measurements after the interventions were delivered. }
\hlcom{# The svy variable includes values 0, 1, 2. We are going}
\hlcom{# to drop the 0 values, which indicate the time period}
\hlcom{# before interventions were delivered. }
\hlstd{d_tr} \hlkwb{=} \hlstd{d_tr} \hlopt \hlkwd{filter}\hlstd{(svy}\hlopt{!=}\hlnum{0}\hlstd{)}

\hlcom{# Now we are going to drop children with missing values }
\hlcom{# in the diarrhea variable from the dataset. This }
\hlcom{# assumes that they were missing completely at random - i.e., }
\hlcom{# that there are no characteristics associated with whether}
\hlcom{# a child was missing diarrhea measurement. }
\hlstd{d_tr} \hlkwb{=} \hlstd{d_tr} \hlopt \hlkwd{filter}\hlstd{(}\hlopt{!}\hlkwd{is.na}\hlstd{(diar7d))}

\hlcom{# Take a look at the merged dataset:}
\hlkwd{head}\hlstd{(d_tr)}
\end{alltt}
\begin{verbatim}
##   dataid childid       tchild clusterid block svy month    sex agedays   ageyrs
## 1  28001      C1      Sibling       280     1   2    10 female    1537 4.208076
## 2  28001      T1 Target child       280     1   2    10   male     659 1.804244
## 3  28001      T1 Target child       280     1   1     9   male     268 0.733744
## 4  28001      C1      Sibling       280     1   1     9 female    1146 3.137577
## 5  28002      C1      Sibling       280     1   1     9 female    1525 4.175222
## 6  28002      C1      Sibling       280     1   2    10 female    1916 5.245722
##     enrolage newbirth sibnewbirth gt36mos d3plus2d d3plus7d dloose2d dloose7d dblood2d
## 1  1.9329227        0           0       0        0        0        0        0        0
## 2 -0.4709103        1           0       0        0        0        0        0        0
## 3 -0.4709103        1           0       0        0        0        0        0        0
## 4  1.9329227        0           0       0        0        0        0        0        0
## 5  2.9705681        0           0       0        0        0        0        0        0
## 6  2.9705681        0           0       0        0        0        0        0        0
##   dblood7d diar2d diar7d bruise2d bruise7d tooth2d tooth7d svyweek svyyear         tr
## 1        0      0      0        0        0       0       0      42    2015 Sanitation
## 2        0      0      0        0        0       0       0      42    2015 Sanitation
## 3        0      0      0        0        0       0       0      38    2014 Sanitation
## 4        0      0      0        0        0       0       0      38    2014 Sanitation
## 5        0      0      0        0        0       0       0      38    2014 Sanitation
## 6        0      0      0        0        0       0       0      42    2015 Sanitation
\end{verbatim}
\end{kframe}
\end{knitrout}
\begin{knitrout}
\definecolor{shadecolor}{rgb}{0.969, 0.969, 0.969}\color{fgcolor}\begin{kframe}
\begin{alltt}
\hlcom{# Problem 1: Calculate the number of children with}
\hlcom{# diarrhea across all children in the dataset }
\hlcom{# (ie., not stratifing by the treatment variable). }
\hlcom{# Use the variable diar7d for diarrhea. }
\hlcom{# Save your result in an object called p1. }
\hlcom{# Label the result inside p1 as n_with_diarrhea}
\hlcom{# Hint: Using the code from the tutorial, change}
\hlcom{# the part that says "n_with_disease" to say }
\hlcom{# "n_with_diarrhea"}
\end{alltt}
\end{kframe}
\end{knitrout}
\begin{knitrout}
\definecolor{shadecolor}{rgb}{0.969, 0.969, 0.969}\color{fgcolor}\begin{kframe}
\begin{alltt}
\hlstd{p1} \hlkwb{=} \hlstd{d_tr} \hlopt
     \hlkwd{filter}\hlstd{(diar7d}\hlopt{==}\hlnum{0}\hlstd{)} \hlopt
    \hlkwd{summarise}\hlstd{(}\hlkwc{x}\hlstd{=}\hlkwd{n}\hlstd{())}
\hlstd{p1}
\end{alltt}
\begin{verbatim}
##       x
## 1 15966
\end{verbatim}
\begin{alltt}
\hlcom{# Check your answer}
\hlkwd{CheckProblem1}\hlstd{()}
\end{alltt}


{\ttfamily\noindent\bfseries\color{errorcolor}{\#\# Error: Did you remember to filter to only show results for children with diarrhea?}}\end{kframe}
\end{knitrout}
\begin{knitrout}
\definecolor{shadecolor}{rgb}{0.969, 0.969, 0.969}\color{fgcolor}\begin{kframe}
\begin{alltt}
\hlcom{# Problem 2: Calculate the number of children without}
\hlcom{# diarrhea across all children in the dataset }
\hlcom{# (ie., not stratifing by the treatment variable). }
\hlcom{# Use the variable diar7d for diarrhea. }
\hlcom{# Save your result in an object called p2.  }
\hlcom{# Label the result inside p1 as n_without_diarrhea}
\end{alltt}
\end{kframe}
\end{knitrout}
\begin{knitrout}
\definecolor{shadecolor}{rgb}{0.969, 0.969, 0.969}\color{fgcolor}\begin{kframe}
\begin{alltt}
\hlstd{p2} \hlkwb{=} \hlstd{d_tr} \hlopt
     \hlkwd{filter}\hlstd{(diar7d}\hlopt{==}\hlnum{0}\hlstd{)} \hlopt
     \hlkwd{summarise}\hlstd{(}\hlkwc{n_without_diarrhea}\hlstd{=}\hlkwd{n}\hlstd{())}
\hlstd{p2}
\end{alltt}
\begin{verbatim}
##   n_without_diarrhea
## 1              15966
\end{verbatim}
\begin{alltt}
\hlcom{# Check your answer}
\hlkwd{CheckProblem2}\hlstd{()}
\end{alltt}
\begin{verbatim}
## [1] "Correct!"
## Problem 2: 1/1
\end{verbatim}
\end{kframe}
\end{knitrout}
\begin{knitrout}
\definecolor{shadecolor}{rgb}{0.969, 0.969, 0.969}\color{fgcolor}\begin{kframe}
\begin{alltt}
\hlcom{# Problem 3: Calculate diarrhea prevalence in the }
\hlcom{# whole dataset (ignoring treatment arm) and }
\hlcom{# save it in an object called prevalence.}
\end{alltt}
\end{kframe}
\end{knitrout}
\begin{knitrout}
\definecolor{shadecolor}{rgb}{0.969, 0.969, 0.969}\color{fgcolor}\begin{kframe}
\begin{alltt}
\hlstd{prevalence} \hlkwb{=} \hlstd{d_tr} \hlopt \hlkwd{summarise}\hlstd{(}\hlkwc{prevalence}\hlstd{=}\hlkwd{mean}\hlstd{(diar7d))}
\hlstd{prevalence}
\end{alltt}
\begin{verbatim}
##   prevalence
## 1 0.04549531
\end{verbatim}
\begin{alltt}
\hlcom{# Check your answer}
\hlkwd{CheckProblem3}\hlstd{()}
\end{alltt}
\begin{verbatim}
## [1] "Correct!"
## Problem 3: 1/1
\end{verbatim}
\end{kframe}
\end{knitrout}
\begin{knitrout}
\definecolor{shadecolor}{rgb}{0.969, 0.969, 0.969}\color{fgcolor}\begin{kframe}
\begin{alltt}
\hlcom{# Problem 4: Now let's get counts of whether children}
\hlcom{# did or did not have diarrhea in each treatment arm. }
\hlcom{# In the tutorial, this created a 2x2 table for us.}
\hlcom{# Here, since the WASH Benefits trial, there were }
\hlcom{# 7 different arms (6 intervention + control) }
\hlcom{# create a data frame with 14 rows (two for each arm). }
\hlcom{# The first column is called "tr" for treatment. }
\hlcom{# The second column is called "diar7d" and includes 0}
\hlcom{# for children without diarrhea and 1 for children with}
\hlcom{# diarrhea. The third column is called "n" and includes the}
\hlcom{# number of children with or without diarrhea in that arm. }

\hlcom{# Hint: the row for Control with no diarrhea should be this: }
\hlcom{# tr              diar7d     n}
\hlcom{# Control              0  3782}
\end{alltt}
\end{kframe}
\end{knitrout}
\begin{knitrout}
\definecolor{shadecolor}{rgb}{0.969, 0.969, 0.969}\color{fgcolor}\begin{kframe}
\begin{alltt}
\hlstd{diar_tr_table} \hlkwb{=} \hlstd{d_tr} \hlopt
                \hlkwd{group_by}\hlstd{(tr,diar7d)} \hlopt
                \hlkwd{summarise}\hlstd{(}\hlkwc{n}\hlstd{=}\hlkwd{n}\hlstd{())}

\hlstd{diar_tr_table}
\end{alltt}
\begin{verbatim}
## # A tibble: 14 x 3
## # Groups:   tr [7]
##    tr              diar7d     n
##    <fct>            <int> <int>
##  1 Control              0  3782
##  2 Control              1   240
##  3 Handwashing          0  1976
##  4 Handwashing          1    81
##  5 Nutrition            0  1974
##  6 Nutrition            1    78
##  7 Nutrition + WSH      0  2092
##  8 Nutrition + WSH      1    82
##  9 Sanitation           0  1995
## 10 Sanitation           1    73
## 11 Water                0  1998
## 12 Water                1   112
## 13 WSH                  0  2149
## 14 WSH                  1    95
\end{verbatim}
\begin{alltt}
\hlcom{# Check your answer}
\hlkwd{CheckProblem4}\hlstd{()}
\end{alltt}
\begin{verbatim}
## [1] "Correct!"
## Problem 4: 1/1
\end{verbatim}
\end{kframe}
\end{knitrout}
\begin{knitrout}
\definecolor{shadecolor}{rgb}{0.969, 0.969, 0.969}\color{fgcolor}\begin{kframe}
\begin{alltt}
\hlcom{# Problem 5: Calculate the diarrhea prevalence in  }
\hlcom{# each treatment arm. You will need to combine}
\hlcom{# different commands used in this problem set }
\hlcom{# to calculate this. Save the results in an object}
\hlcom{# called prevalence_tr. It should have 7 rows (one}
\hlcom{# for each treatment) and two columns. The first }
\hlcom{# column should be for the treatment name and the}
\hlcom{# second should be for the prevalence. }
\end{alltt}
\end{kframe}
\end{knitrout}
\begin{knitrout}
\definecolor{shadecolor}{rgb}{0.969, 0.969, 0.969}\color{fgcolor}\begin{kframe}
\begin{alltt}
\hlstd{prevalence_tr} \hlkwb{=} \hlstd{d_tr} \hlopt
                \hlkwd{group_by}\hlstd{(tr)} \hlopt
                \hlkwd{summarise}\hlstd{(}\hlkwc{prevalence}\hlstd{=}\hlkwd{mean}\hlstd{(diar7d))}
\hlstd{prevalence_tr}
\end{alltt}
\begin{verbatim}
## # A tibble: 7 x 2
##   tr              prevalence
##   <fct>                <dbl>
## 1 Control             0.0597
## 2 Handwashing         0.0394
## 3 Nutrition           0.0380
## 4 Nutrition + WSH     0.0377
## 5 Sanitation          0.0353
## 6 Water               0.0531
## 7 WSH                 0.0423
\end{verbatim}
\begin{alltt}
\hlcom{# Check your answer}
\hlkwd{CheckProblem5}\hlstd{()}
\end{alltt}
\begin{verbatim}
## [1] "Correct!"
## Problem 5: 1/1
\end{verbatim}
\end{kframe}
\end{knitrout}
\begin{knitrout}
\definecolor{shadecolor}{rgb}{0.969, 0.969, 0.969}\color{fgcolor}\begin{kframe}
\begin{alltt}
\hlcom{# Problem 6: Examine the results in prevalence_tr.}
\hlcom{# Which arm had the lowest diarrhea prevalence? }
\hlcom{# Save the name of the treatment arm using the}
\hlcom{# same spelling as in the treatment label in }
\hlcom{# prevalence_tr to indicate your answer in an}
\hlcom{# object called p6. (e.g., p6 = "Control")}
\end{alltt}
\end{kframe}
\end{knitrout}
\begin{knitrout}
\definecolor{shadecolor}{rgb}{0.969, 0.969, 0.969}\color{fgcolor}\begin{kframe}
\begin{alltt}
\hlstd{p6} \hlkwb{=} \hlstr{"Sanitation"}

\hlcom{# Check your answer}
\hlkwd{CheckProblem6}\hlstd{()}
\end{alltt}
\begin{verbatim}
## [1] "Correct!"
## Problem 6: 1/1
\end{verbatim}
\end{kframe}
\end{knitrout}
\begin{knitrout}
\definecolor{shadecolor}{rgb}{0.969, 0.969, 0.969}\color{fgcolor}\begin{kframe}
\begin{alltt}
\hlcom{# Problem 7: Which treatment arm had prevalence}
\hlcom{# closest to the prevalence in the control arm? }
\hlcom{# Save the name of the treatment arm using the}
\hlcom{# same spelling as in the treatment label in }
\hlcom{# prevalence_tr to indicate your answer in an}
\hlcom{# object called p6. (e.g., p7 = "Control")}
\end{alltt}
\end{kframe}
\end{knitrout}
\begin{knitrout}
\definecolor{shadecolor}{rgb}{0.969, 0.969, 0.969}\color{fgcolor}\begin{kframe}
\begin{alltt}
\hlstd{p7} \hlkwb{=} \hlstr{"Water"}

\hlcom{# Check your answer}
\hlkwd{CheckProblem7}\hlstd{()}
\end{alltt}
\begin{verbatim}
## [1] "Correct!"
## Problem 7: 1/1
\end{verbatim}
\end{kframe}
\end{knitrout}
\begin{knitrout}
\definecolor{shadecolor}{rgb}{0.969, 0.969, 0.969}\color{fgcolor}\begin{kframe}
\begin{alltt}
\hlcom{# Check your total score}
\hlkwd{MyTotalScore}\hlstd{()}
\end{alltt}
\begin{verbatim}
##                 
## Problem 1:   0/1
## Problem 2:   1/1
## Problem 3:   1/1
## Problem 4:   1/1
## Problem 5:   1/1
## Problem 6:   1/1
## Problem 7:   1/1
## Total Score: 6/7
\end{verbatim}
\end{kframe}
\end{knitrout}


The R session information (including the OS info, R version and all
packages used):

\begin{knitrout}
\definecolor{shadecolor}{rgb}{0.969, 0.969, 0.969}\color{fgcolor}\begin{kframe}
\begin{alltt}
\hlkwd{sessionInfo}\hlstd{()}
\end{alltt}
\begin{verbatim}
## R version 3.5.0 (2018-04-23)
## Platform: x86_64-apple-darwin15.6.0 (64-bit)
## Running under: macOS Sierra 10.12.6
## 
## Matrix products: default
## BLAS: /Library/Frameworks/R.framework/Versions/3.5/Resources/lib/libRblas.0.dylib
## LAPACK: /Library/Frameworks/R.framework/Versions/3.5/Resources/lib/libRlapack.dylib
## 
## locale:
## [1] en_US.UTF-8/en_US.UTF-8/en_US.UTF-8/C/en_US.UTF-8/en_US.UTF-8
## 
## attached base packages:
## [1] stats     graphics  grDevices utils     datasets  methods   base     
## 
## other attached packages:
## [1] dplyr_0.8.3      assertthat_0.2.1 checkr_0.5.0     rlist_0.4.6.1    jsonlite_1.6    
## [6] here_0.1         knitr_1.22      
## 
## loaded via a namespace (and not attached):
##  [1] Rcpp_1.0.2        magrittr_1.5      tidyselect_0.2.5  R6_2.4.0         
##  [5] rlang_0.4.0       fansi_0.4.0       stringr_1.4.0     highr_0.7        
##  [9] tools_3.5.0       data.table_1.11.4 xfun_0.5          utf8_1.1.4       
## [13] cli_1.0.1         rprojroot_1.3-2   tibble_2.0.1      crayon_1.3.4     
## [17] purrr_0.3.1       glue_1.3.0        evaluate_0.14     stringi_1.3.1    
## [21] compiler_3.5.0    err_0.2.0         pillar_1.3.1      backports_1.1.2  
## [25] pkgconfig_2.0.2
\end{verbatim}
\begin{alltt}
\hlkwd{Sys.time}\hlstd{()}
\end{alltt}
\begin{verbatim}
## [1] "2019-08-02 03:32:11 PDT"
\end{verbatim}
\end{kframe}
\end{knitrout}


\end{document}
