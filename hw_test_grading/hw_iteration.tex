\documentclass{article}\usepackage[]{graphicx}\usepackage[]{color}
%% maxwidth is the original width if it is less than linewidth
%% otherwise use linewidth (to make sure the graphics do not exceed the margin)
\makeatletter
\def\maxwidth{ %
  \ifdim\Gin@nat@width>\linewidth
    \linewidth
  \else
    \Gin@nat@width
  \fi
}
\makeatother

\definecolor{fgcolor}{rgb}{0.345, 0.345, 0.345}
\newcommand{\hlnum}[1]{\textcolor[rgb]{0.686,0.059,0.569}{#1}}%
\newcommand{\hlstr}[1]{\textcolor[rgb]{0.192,0.494,0.8}{#1}}%
\newcommand{\hlcom}[1]{\textcolor[rgb]{0.678,0.584,0.686}{\textit{#1}}}%
\newcommand{\hlopt}[1]{\textcolor[rgb]{0,0,0}{#1}}%
\newcommand{\hlstd}[1]{\textcolor[rgb]{0.345,0.345,0.345}{#1}}%
\newcommand{\hlkwa}[1]{\textcolor[rgb]{0.161,0.373,0.58}{\textbf{#1}}}%
\newcommand{\hlkwb}[1]{\textcolor[rgb]{0.69,0.353,0.396}{#1}}%
\newcommand{\hlkwc}[1]{\textcolor[rgb]{0.333,0.667,0.333}{#1}}%
\newcommand{\hlkwd}[1]{\textcolor[rgb]{0.737,0.353,0.396}{\textbf{#1}}}%
\let\hlipl\hlkwb

\usepackage{framed}
\makeatletter
\newenvironment{kframe}{%
 \def\at@end@of@kframe{}%
 \ifinner\ifhmode%
  \def\at@end@of@kframe{\end{minipage}}%
  \begin{minipage}{\columnwidth}%
 \fi\fi%
 \def\FrameCommand##1{\hskip\@totalleftmargin \hskip-\fboxsep
 \colorbox{shadecolor}{##1}\hskip-\fboxsep
     % There is no \\@totalrightmargin, so:
     \hskip-\linewidth \hskip-\@totalleftmargin \hskip\columnwidth}%
 \MakeFramed {\advance\hsize-\width
   \@totalleftmargin\z@ \linewidth\hsize
   \@setminipage}}%
 {\par\unskip\endMakeFramed%
 \at@end@of@kframe}
\makeatother

\definecolor{shadecolor}{rgb}{.97, .97, .97}
\definecolor{messagecolor}{rgb}{0, 0, 0}
\definecolor{warningcolor}{rgb}{1, 0, 1}
\definecolor{errorcolor}{rgb}{1, 0, 0}
\newenvironment{knitrout}{}{} % an empty environment to be redefined in TeX

\usepackage{alltt}
\usepackage[sc]{mathpazo}
\renewcommand{\sfdefault}{lmss}
\renewcommand{\ttdefault}{lmtt}
\usepackage[T1]{fontenc}
\usepackage{geometry}
\geometry{verbose,tmargin=2.5cm,bmargin=2.5cm,lmargin=2.5cm,rmargin=2.5cm}
\setcounter{secnumdepth}{2}
\setcounter{tocdepth}{2}
\usepackage[unicode=true,pdfusetitle,
 bookmarks=true,bookmarksnumbered=true,bookmarksopen=true,bookmarksopenlevel=2,
 breaklinks=false,pdfborder={0 0 1},backref=false,colorlinks=false]
 {hyperref}
\hypersetup{
 pdfstartview={XYZ null null 1}}

\makeatletter
%%%%%%%%%%%%%%%%%%%%%%%%%%%%%% User specified LaTeX commands.
\renewcommand{\textfraction}{0.05}
\renewcommand{\topfraction}{0.8}
\renewcommand{\bottomfraction}{0.8}
\renewcommand{\floatpagefraction}{0.75}

\makeatother
\IfFileExists{upquote.sty}{\usepackage{upquote}}{}
\begin{document}








The results below are generated from an R script.

\begin{knitrout}
\definecolor{shadecolor}{rgb}{0.969, 0.969, 0.969}\color{fgcolor}\begin{kframe}
\begin{alltt}
\hlcom{#####################################}
\hlcom{# Epidemiologic Methods II }
\hlcom{# PHW250F, PHW250G, PH250B}

\hlcom{# Solutions: Iteration}
\hlcom{#####################################}
\hlcom{# Load okR autograder}
\hlkwd{library}\hlstd{(here)}
\end{alltt}


{\ttfamily\noindent\itshape\color{messagecolor}{\#\# here() starts at /Users/Nolan}}\begin{alltt}
\hlkwd{source}\hlstd{(}\hlstr{"hw_iteration.ok.R"}\hlstd{)}
\end{alltt}


{\ttfamily\noindent\color{warningcolor}{\#\# Warning: package 'dplyr' was built under R version 3.5.2}}

{\ttfamily\noindent\itshape\color{messagecolor}{\#\# \\\#\# Attaching package: 'dplyr'}}

{\ttfamily\noindent\itshape\color{messagecolor}{\#\# The following objects are masked from 'package:stats':\\\#\# \\\#\#\ \ \ \  filter, lag}}

{\ttfamily\noindent\itshape\color{messagecolor}{\#\# The following objects are masked from 'package:base':\\\#\# \\\#\#\ \ \ \  intersect, setdiff, setequal, union}}

{\ttfamily\noindent\color{warningcolor}{\#\# Warning: package 'reticulate' was built under R version 3.5.2}}\begin{alltt}
\hlkwd{AutograderInit}\hlstd{()}

\hlcom{# Let's load the dplyr package}
\hlkwd{library}\hlstd{(dplyr)}

\hlcom{# This assignment uses data from the WASH Benefits}
\hlcom{# Bangladesh trial. The trial assessed whether }
\hlcom{# water, sanitation, handwashing, and nutrition }
\hlcom{# interventions delivered separately or together}
\hlcom{# could reduce child diarrhea and/or improve child}
\hlcom{# growth. See Luby et al. 2018 for full details }
\hlcom{# (doi: http://dx.doi.org/10.1016/)}

\hlcom{# In this problem set we will calculate the incidence }
\hlcom{# density of diarrhea in different treatment arms after the }
\hlcom{# interventions were delivered. }

\hlcom{# The data and codebooks are publicly available here: }
\hlcom{# https://osf.io/pqzj5/}

\hlcom{# Make sure you downloaded the data from }
\hlcom{# the course site. Save the csv files in the same}
\hlcom{# location on your computer as the script hw_iteration.R}
\hlcom{# Then use the following commands to read in the data. }

\hlcom{# Load the diarrhea dataset:}
\hlstd{d} \hlkwb{=} \hlkwd{read.csv}\hlstd{(}\hlstr{"washb-bangladesh-diar-public.csv"}\hlstd{)}

\hlcom{# Load the dataset with treatment variables: }
\hlstd{tr} \hlkwb{=} \hlkwd{read.csv}\hlstd{(}\hlstr{"washb-bangladesh-tr-public.csv"}\hlstd{)}

\hlcom{# Next let's merge the two datasets together.}
\hlcom{# This will allow us to calculate the prevalence}
\hlcom{# of diarrhea in different treatment arms }
\hlcom{# (e.g., water, sanitation, handwashing, etc.)}

\hlstd{d_tr} \hlkwb{=} \hlkwd{left_join}\hlstd{(d, tr,} \hlkwc{by}\hlstd{=}\hlkwd{c}\hlstd{(}\hlstr{"block"}\hlstd{,}\hlstr{"clusterid"}\hlstd{))}

\hlcom{# Now let's filter to only keep the rows for diarrhea}
\hlcom{# measurements after the interventions were delivered. }
\hlcom{# The svy variable includes values 0, 1, 2. We are going}
\hlcom{# to drop the 0 values, which indicate the time period}
\hlcom{# before interventions were delivered. }
\hlstd{d_tr} \hlkwb{=} \hlstd{d_tr} \hlopt \hlkwd{filter}\hlstd{(svy}\hlopt{!=}\hlnum{0}\hlstd{)}

\hlcom{# Now we are going to drop children with missing values }
\hlcom{# in the diarrhea variable from the dataset. This }
\hlcom{# assumes that they were missing at random - i.e., }
\hlcom{# that there are no characteristics associated with whether}
\hlcom{# a child was missing diarrhea measurement. }
\hlstd{d_tr} \hlkwb{=} \hlstd{d_tr} \hlopt \hlkwd{filter}\hlstd{(}\hlopt{!}\hlkwd{is.na}\hlstd{(diar7d))}

\hlcom{# Take a look at the merged dataset:}
\hlkwd{head}\hlstd{(d_tr)}
\end{alltt}
\begin{verbatim}
##   dataid childid       tchild clusterid block svy month    sex agedays   ageyrs
## 1  28001      C1      Sibling       280     1   2    10 female    1537 4.208076
## 2  28001      T1 Target child       280     1   2    10   male     659 1.804244
## 3  28001      T1 Target child       280     1   1     9   male     268 0.733744
## 4  28001      C1      Sibling       280     1   1     9 female    1146 3.137577
## 5  28002      C1      Sibling       280     1   1     9 female    1525 4.175222
## 6  28002      C1      Sibling       280     1   2    10 female    1916 5.245722
##     enrolage newbirth sibnewbirth gt36mos d3plus2d d3plus7d dloose2d dloose7d dblood2d
## 1  1.9329227        0           0       0        0        0        0        0        0
## 2 -0.4709103        1           0       0        0        0        0        0        0
## 3 -0.4709103        1           0       0        0        0        0        0        0
## 4  1.9329227        0           0       0        0        0        0        0        0
## 5  2.9705681        0           0       0        0        0        0        0        0
## 6  2.9705681        0           0       0        0        0        0        0        0
##   dblood7d diar2d diar7d bruise2d bruise7d tooth2d tooth7d svyweek svyyear         tr
## 1        0      0      0        0        0       0       0      42    2015 Sanitation
## 2        0      0      0        0        0       0       0      42    2015 Sanitation
## 3        0      0      0        0        0       0       0      38    2014 Sanitation
## 4        0      0      0        0        0       0       0      38    2014 Sanitation
## 5        0      0      0        0        0       0       0      38    2014 Sanitation
## 6        0      0      0        0        0       0       0      42    2015 Sanitation
\end{verbatim}
\end{kframe}
\end{knitrout}
\begin{knitrout}
\definecolor{shadecolor}{rgb}{0.969, 0.969, 0.969}\color{fgcolor}\begin{kframe}
\begin{alltt}
\hlcom{# Problem 1: Calculate the diarrhea prevalence in  }
\hlcom{# each treatment arm. Save the results in an object}
\hlcom{# called prevalence_tr. It should have 7 rows (one}
\hlcom{# for each treatment) and two columns. The first }
\hlcom{# column should be for the treatment name and the}
\hlcom{# second should be for the prevalence. }

\hlcom{# (Hint: you did this in the homework assignment }
\hlcom{# called hw_prev, and you can use the same code)}
\end{alltt}
\end{kframe}
\end{knitrout}
\begin{knitrout}
\definecolor{shadecolor}{rgb}{0.969, 0.969, 0.969}\color{fgcolor}\begin{kframe}
\begin{alltt}
\hlstd{prevalence_tr} \hlkwb{=} \hlstd{d_tr} \hlopt
  \hlkwd{group_by}\hlstd{(tr)} \hlopt
  \hlkwd{summarise}\hlstd{(}\hlkwc{prevalence}\hlstd{=}\hlkwd{mean}\hlstd{(diar7d))}
\hlstd{prevalence_tr}
\end{alltt}
\begin{verbatim}
## # A tibble: 7 x 2
##   tr              prevalence
##   <fct>                <dbl>
## 1 Control             0.0597
## 2 Handwashing         0.0394
## 3 Nutrition           0.0380
## 4 Nutrition + WSH     0.0377
## 5 Sanitation          0.0353
## 6 Water               0.0531
## 7 WSH                 0.0423
\end{verbatim}
\begin{alltt}
\hlcom{# Check your answer}
\hlkwd{CheckProblem1}\hlstd{()}
\end{alltt}
\begin{verbatim}
## [1] "Correct!!"
## Problem 1: 1/1
\end{verbatim}
\end{kframe}
\end{knitrout}
\begin{knitrout}
\definecolor{shadecolor}{rgb}{0.969, 0.969, 0.969}\color{fgcolor}\begin{kframe}
\begin{alltt}
\hlcom{# Problem 2: Write a function that converts}
\hlcom{# prevalence to incidence density. Assume }
\hlcom{# that the disease is rare. The function should}
\hlcom{# be called calculate_inc_rare. It should take}
\hlcom{# two arguments for prevalence (prev) and duration (d)}
\hlcom{# of disease. It should return the estimated}
\hlcom{# incidence density in person-days. }
\end{alltt}
\end{kframe}
\end{knitrout}
\begin{knitrout}
\definecolor{shadecolor}{rgb}{0.969, 0.969, 0.969}\color{fgcolor}\begin{kframe}
\begin{alltt}
\hlstd{calculate_inc_rare}\hlkwb{=}\hlkwa{function}\hlstd{(}\hlkwc{prev}\hlstd{,}\hlkwc{d}\hlstd{)\{}
  \hlstd{id} \hlkwb{=} \hlstd{prev} \hlopt{/} \hlstd{d}
  \hlkwd{return}\hlstd{(id)}
\hlstd{\}}

\hlcom{# Check your answer}
\hlcom{# Note: the autograder will only check that}
\hlcom{# you defined a function with the correct name}
\hlcom{# and that the arguments are named correctly. }
\hlkwd{CheckProblem2}\hlstd{()}
\end{alltt}
\begin{verbatim}
## [1] "Correct!"
## Problem 2: 1/1
\end{verbatim}
\end{kframe}
\end{knitrout}
\begin{knitrout}
\definecolor{shadecolor}{rgb}{0.969, 0.969, 0.969}\color{fgcolor}\begin{kframe}
\begin{alltt}
\hlcom{# Problem 3: Using dplyr (and not a foor loop), }
\hlcom{# add a column to the data frame}
\hlcom{# prevalence_tr that has the incidence density}
\hlcom{# in each arm assuming that the duration is 5 days.}
\hlcom{# Use the function calculate_inc_rare to }
\hlcom{# estimate the incidence density and name the column }
\hlcom{# incidence.}
\end{alltt}
\end{kframe}
\end{knitrout}
\begin{knitrout}
\definecolor{shadecolor}{rgb}{0.969, 0.969, 0.969}\color{fgcolor}\begin{kframe}
\begin{alltt}
\hlstd{prevalence_tr} \hlkwb{=} \hlstd{prevalence_tr} \hlopt
  \hlkwd{mutate}\hlstd{(}\hlkwc{incidence} \hlstd{=} \hlkwd{calculate_inc_rare}\hlstd{(}\hlkwc{prev} \hlstd{= prevalence,} \hlkwc{d} \hlstd{=} \hlnum{5}\hlstd{))}

\hlcom{# Check your answer}
\hlkwd{CheckProblem3}\hlstd{()}
\end{alltt}
\begin{verbatim}
## [1] "Correct!"
## Problem 3: 1/1
\end{verbatim}
\end{kframe}
\end{knitrout}
\begin{knitrout}
\definecolor{shadecolor}{rgb}{0.969, 0.969, 0.969}\color{fgcolor}\begin{kframe}
\begin{alltt}
\hlcom{# Problem 4: Using dplyr (and not a for loop), }
\hlcom{# calculate the incidence density difference}
\hlcom{# (incidence density in each intervention arm minus }
\hlcom{# the incidence density in the control arm). }
\hlcom{# Save results in a column called inc_diff }
\hlcom{# in the data frame prevalence_tr. }
\end{alltt}
\end{kframe}
\end{knitrout}
\begin{knitrout}
\definecolor{shadecolor}{rgb}{0.969, 0.969, 0.969}\color{fgcolor}\begin{kframe}
\begin{alltt}
\hlcom{# To get you started, here is code to save}
\hlcom{# the incidence density in the control arm as }
\hlcom{# a scalar. You can subtract this scalar from the}
\hlcom{# incidence density in each treatment arm. }
\hlstd{control_inc} \hlkwb{=} \hlstd{prevalence_tr} \hlopt
  \hlkwd{filter}\hlstd{(tr}\hlopt{==}\hlstr{"Control"}\hlstd{)} \hlopt
  \hlkwd{select}\hlstd{(incidence)}
\hlstd{control_inc} \hlkwb{=} \hlkwd{as.vector}\hlstd{(control_inc}\hlopt{$}\hlstd{incidence)}

\hlcom{# calculate the incidence difference in each arm }
\hlstd{prevalence_tr} \hlkwb{=} \hlstd{prevalence_tr} \hlopt
  \hlkwd{mutate}\hlstd{(}\hlkwc{inc_diff} \hlstd{= incidence} \hlopt{-} \hlstd{control_inc)}

\hlcom{# Check your answer}
\hlkwd{CheckProblem4}\hlstd{()}
\end{alltt}
\begin{verbatim}
## [1] "Correct!"
## Problem 4: 1/1
\end{verbatim}
\end{kframe}
\end{knitrout}
\begin{knitrout}
\definecolor{shadecolor}{rgb}{0.969, 0.969, 0.969}\color{fgcolor}\begin{kframe}
\begin{alltt}
\hlcom{# Problem 5: Using a for loop (and not dplyr), }
\hlcom{# calculate the incidence density difference}
\hlcom{# (incidence density in each intervention arm minus }
\hlcom{# the incidence density in the control arm). }
\hlcom{# Save results in a vector called inc_diff_loop.}
\hlcom{# Hint: the length of the vector should be 6.}
\hlcom{# Do not include the control arm in the vector.}

\hlcom{# The purpose of this problem is to show you }
\hlcom{# that there is more than one way to do the same}
\hlcom{# thing in R. This is an example of a situation}
\hlcom{# in which most R users would probably prefer to }
\hlcom{# use dplyr, but for your learning purposes, it is}
\hlcom{# helpful to practice using a loop to see the pros}
\hlcom{# and cons of each approach. }
\end{alltt}
\end{kframe}
\end{knitrout}
\begin{knitrout}
\definecolor{shadecolor}{rgb}{0.969, 0.969, 0.969}\color{fgcolor}\begin{kframe}
\begin{alltt}
\hlstd{inc_diff_loop} \hlkwb{=} \hlkwd{vector}\hlstd{(}\hlkwc{length}\hlstd{=}\hlnum{6}\hlstd{)}
\hlkwa{for}\hlstd{(i} \hlkwa{in} \hlnum{1}\hlopt{:}\hlkwd{length}\hlstd{(inc_diff_loop))\{}
  \hlstd{inc_diff_loop[i]} \hlkwb{=} \hlstd{prevalence_tr}\hlopt{$}\hlstd{incidence[i}\hlopt{+}\hlnum{1}\hlstd{]} \hlopt{-} \hlstd{control_inc}
\hlstd{\}}

\hlstd{inc_diff_loop}
\end{alltt}
\begin{verbatim}
## [1] -0.004058814 -0.004332022 -0.004390663 -0.004874400 -0.001318247 -0.003467338
\end{verbatim}
\begin{alltt}
\hlcom{# Check your answer}
\hlkwd{CheckProblem5}\hlstd{()}
\end{alltt}
\begin{verbatim}
## [1] "Correct!"
## Problem 5: 1/1
\end{verbatim}
\end{kframe}
\end{knitrout}
\begin{knitrout}
\definecolor{shadecolor}{rgb}{0.969, 0.969, 0.969}\color{fgcolor}\begin{kframe}
\begin{alltt}
\hlcom{# Problem 6: Optional challenge question!}

\hlcom{# In the data frame d_tr, columns d3plus2d through tooth7d }
\hlcom{# contain indicators for whether each child}
\hlcom{# experienced those symptoms in the past 7 days}
\hlcom{# or the past 2 days. }

\hlcom{# Write a for loop to take the mean of each of those }
\hlcom{# columns, and save the results in a vector called}
\hlcom{# symptoms. When taking the mean, you will need to }
\hlcom{# use the option na.rm=TRUE in order to ignore}
\hlcom{# missing values when calculating the mean.}

\hlcom{# There is more than one way to approach this problem. }
\end{alltt}
\end{kframe}
\end{knitrout}
\begin{knitrout}
\definecolor{shadecolor}{rgb}{0.969, 0.969, 0.969}\color{fgcolor}\begin{kframe}
\begin{alltt}
\hlstd{symptoms_df} \hlkwb{=} \hlstd{d_tr} \hlopt
  \hlkwd{select}\hlstd{(d3plus2d}\hlopt{:}\hlstd{tooth7d)}

\hlstd{symptoms} \hlkwb{=} \hlkwd{vector}\hlstd{(}\hlkwc{length} \hlstd{=} \hlkwd{ncol}\hlstd{(symptoms_df))}
\hlkwa{for}\hlstd{(i} \hlkwa{in} \hlnum{1}\hlopt{:}\hlkwd{length}\hlstd{(symptoms))\{}
  \hlstd{symptoms[i]} \hlkwb{=} \hlkwd{mean}\hlstd{(symptoms_df[,i],} \hlkwc{na.rm}\hlstd{=}\hlnum{TRUE}\hlstd{)}
\hlstd{\}}

\hlcom{# Check your answer}
\hlkwd{CheckProblem6}\hlstd{()}
\end{alltt}
\begin{verbatim}
## [1] "Correct!"
## Problem 6: 1/1
\end{verbatim}
\end{kframe}
\end{knitrout}
\begin{knitrout}
\definecolor{shadecolor}{rgb}{0.969, 0.969, 0.969}\color{fgcolor}\begin{kframe}
\begin{alltt}
\hlcom{# Problem 7: Which symptoms were the most common?}
\hlcom{# Enter your answer as a string scalar named p7}
\hlcom{# including the column name in the quotes.}
\hlcom{# (e.g., p7 = "dloose2d")}
\end{alltt}
\end{kframe}
\end{knitrout}
\begin{knitrout}
\definecolor{shadecolor}{rgb}{0.969, 0.969, 0.969}\color{fgcolor}\begin{kframe}
\begin{alltt}
\hlstd{p7} \hlkwb{=} \hlstr{"d3plus7d"}

\hlcom{# Check your answer}
\hlkwd{CheckProblem7}\hlstd{()}
\end{alltt}
\begin{verbatim}
## [1] "Correct!"
## Problem 7: 1/1
\end{verbatim}
\end{kframe}
\end{knitrout}
\begin{knitrout}
\definecolor{shadecolor}{rgb}{0.969, 0.969, 0.969}\color{fgcolor}\begin{kframe}
\begin{alltt}
\hlcom{# Check your total score}
\hlkwd{MyTotalScore}\hlstd{()}
\end{alltt}
\begin{verbatim}
##                 
## Problem 1:   1/1
## Problem 2:   1/1
## Problem 3:   1/1
## Problem 4:   1/1
## Problem 5:   1/1
## Problem 6:   1/1
## Problem 7:   1/1
## Total Score: 7/7
\end{verbatim}
\end{kframe}
\end{knitrout}


The R session information (including the OS info, R version and all
packages used):

\begin{knitrout}
\definecolor{shadecolor}{rgb}{0.969, 0.969, 0.969}\color{fgcolor}\begin{kframe}
\begin{alltt}
\hlkwd{sessionInfo}\hlstd{()}
\end{alltt}
\begin{verbatim}
## R version 3.5.0 (2018-04-23)
## Platform: x86_64-apple-darwin15.6.0 (64-bit)
## Running under: macOS Sierra 10.12.6
## 
## Matrix products: default
## BLAS: /Library/Frameworks/R.framework/Versions/3.5/Resources/lib/libRblas.0.dylib
## LAPACK: /Library/Frameworks/R.framework/Versions/3.5/Resources/lib/libRlapack.dylib
## 
## locale:
## [1] en_US.UTF-8/en_US.UTF-8/en_US.UTF-8/C/en_US.UTF-8/en_US.UTF-8
## 
## attached base packages:
## [1] stats     graphics  grDevices utils     datasets  methods   base     
## 
## other attached packages:
## [1] reticulate_1.12  dplyr_0.8.0.1    assertthat_0.2.0 checkr_0.3.0     rlist_0.4.6.1   
## [6] jsonlite_1.6     here_0.1         knitr_1.22      
## 
## loaded via a namespace (and not attached):
##  [1] Rcpp_1.0.0        magrittr_1.5      tidyselect_0.2.5  lattice_0.20-35  
##  [5] R6_2.4.0          rlang_0.3.1.9000  fansi_0.4.0       stringr_1.4.0    
##  [9] highr_0.7         tools_3.5.0       grid_3.5.0        data.table_1.11.4
## [13] xfun_0.5          utf8_1.1.4        cli_1.0.1         rprojroot_1.3-2  
## [17] tibble_2.0.1      crayon_1.3.4      Matrix_1.2-17     purrr_0.3.1      
## [21] glue_1.3.0        evaluate_0.13     stringi_1.3.1     compiler_3.5.0   
## [25] pillar_1.3.1      backports_1.1.2   pkgconfig_2.0.2
\end{verbatim}
\begin{alltt}
\hlkwd{Sys.time}\hlstd{()}
\end{alltt}
\begin{verbatim}
## [1] "2019-07-30 00:29:26 PDT"
\end{verbatim}
\end{kframe}
\end{knitrout}


\end{document}
